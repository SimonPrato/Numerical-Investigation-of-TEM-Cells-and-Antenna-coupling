\subsection{Radiated Field}\label{sec:rad_fields}
\FloatBarrier
\subsubsection{Field regions}

\begin{figure}[t]
	\centering
	\includegraphics[width=0.7\linewidth]{content/img/kr_over_distance}
	\caption{Behavior of Expression 1 and Expression 2 in \autoref{eqn:e_et_2} over distance $r$. The distance $r$ is normalized to the radian distance $\lambda / 2\pi$. The magnitude of both expressions is normed to 1 at radian distance for better comparison.}
	\label{fig:kroverdistance}
\end{figure}

The field quantities $\mathbf{E}$ and $\mathbf{H}$ have been derived for an infinitesimal electric dipole in \cref{eqn:e_er,eqn:e_et,eqn:e_ep} and \cref{eqn:e_hr,eqn:e_hpt}, and for an infinitesimal magnetic dipole in \cref{eqn:m_hr,eqn:m_ht,eqn:m_hp} and \cref{eqn:m_ert,eqn:m_ep}. They are valid everywhere except for the source region \cite[p. 156]{Balanis_1997}.

Depending on the distance $r$ to the dipole, the behavior of the fields changes. This becomes apparent when investigating the expressions $1/(jkr)$ and $1/(kr)^2$ in \cref{eqn:e_er,eqn:e_et,eqn:e_ep} and \cref{eqn:e_hr,eqn:e_hpt} of the infinitesimal electric dipole. These expressions are highlighted here in the case of $E_\theta$, although they partly also appear in $E_r$ and $H_\phi$, and referred to as Expression 1 and Expression 2 in

\begin{equation}
	E_\theta = j\eta \frac{kI_0 d \sin \theta }{4\pi r}\left[ 1 + \underbrace{\frac{1}{jkr}}_{\text{Expression 1}} - \underbrace{\frac{1}{(kr)^2}}_{\text{Expression 2}} \right] \mathrm{e}^{-jkr}.
	\label{eqn:e_et_2}
\end{equation}



If the distance $r < \lambda/2\pi$ $(kr < 1)$, then Expression 2 delivers the largest value in the brackets. Consequently, the energy stored in this region is mostly imaginary, especially if $r \ll \lambda/\pi$ ($kr\ll 1$). It is referred to as the near-field region.

At distances $r > \lambda / 2\pi$ $(kr > 1)$, Expression 1 exceeds Expression 2 in value. The real part of the energy is larger than the imaginary part. This region is referred to as the intermediate-field region. For $r \gg \lambda / 2\pi$ ($kr \gg 1$) the energy is primarily real, indicating radiation. This region is called the far-field region.

\begin{figure}[t]
	\centering
	\includegraphics[width=0.4\linewidth]{content/img/field_regions_antenna}
	\caption{Field regions of an antenna, here specifically a linear wire antenna. However, they are applicable for any antenna, as long as their largest dimension $d$ is known.}
	\label{fig:fieldregionsantenna}
\end{figure}

At $r = \lambda/2\pi$ $(kr = 1)$, Expression 1 and Expression 2 are of equal magnitude. This is marked as the radian distance \cite[pp. 156-160]{Balanis_1997}. The radian distance therefore represents an important transition point between field regions, where the behavior of the fields shifts. \autoref{fig:kroverdistance} visualizes Expression 1 and Expression 2 over $r$. The same analysis of the field region is also valid for the infinitesimal magnetic dipole.



%\todo[inline]{The rest of this section may be irrelevant for the thesis: We are interested only in the field regions of an inf. dipole. Maybe leave this for sake of completeness?}

Antennas, which cannot be modeled as infinitesimal dipoles, such as the linear wire antenna, are surrounded by different field regions. They are shown in \autoref{fig:fieldregionsantenna}. The far-field region contains mostly real energy, and the antenna may be most accurately approximated by an infinitesimal electric dipole. In the radiating near-field, the energy is largely real, but depends on the distance $r$. Lastly, in the reactive near-field the energy is mostly imaginary.

The far-field region starts at approximately $r_2$ and the radiating near-field at $r_1$, which are defined as

\begin{subequations}
	\begin{equation}
		r_1=0.62\sqrt{d^3/\lambda},
		\label{eqn:region_r1}
	\end{equation}
	\begin{equation}
	r_2=2d^2/\lambda.
	\label{eqn:region_r2}
	\end{equation}
\end{subequations}

Here, $d$ is the largest dimension of the antenna. In the case of the linear wire antenna, $d$ is the wire length \cite[pp. 165-170]{Balanis_1997}.

\FloatBarrier
\subsubsection{Energy densities and reactances}
The energy density is given by \cite[p. 330]{Griffiths_2024}

\begin{equation}
	w_\mathrm{EM} = \frac{1}{2}\left(\underbrace{\epsilon E^2}_{\text{Electric energy} w_E} + \underbrace{\frac{1}{\mu} B ^ 2}_{\text{Magnetic energy} w_M}\right).
	\label{eqn:em_energy}
\end{equation}

Integrating $w_\mathrm{EM}$ over a volume yields the total electromagnetic energy in this volume. Similarly, integrating $w_E$ gives the electric energy in said volume, and $w_M$ the magnetic energy.

If all of the energy is provided by an electrically short antenna, its equivalent reactances can be derived through \cite[pp 107, 328]{Griffiths_2024}

\begin{subequations}
	\begin{equation}
		L = 2\frac{W_m}{I^2}
		\label{eqn:l_m_energy}
	\end{equation}
	\begin{equation}
		C = 2 \frac{W_e}{V^2}
		\label{eqn:c_e_energy}
	\end{equation}
\end{subequations}

%The electrically short antennas located in the TEM cell interact with it mostly through the reactive near-field region. However, if the TEM cell is large and frequencies are high, the coupling occurs over the radiating near-field. There, the field distributions are different, hence the dipole moments couple differently over frequency to the cell. \todo[inline]{Is this even possible without reaching other modes behavior? And demonstrate the fields over frequency and distance, which changes differently in different regions}.

%An infinitesimal dipole is taken as an antenna. The fields are calculated through the auxiliary fields, as described earlier by \autoref{eqn:elec_and_mag_field_dipole}, which leads to \autoref{eqn:inf_dipole_h_field} for the magnetic field strength and \autoref{eqn:inf_dipole_e_field} for the electric field strength. Those equations are valid everywhere, except for the source point \cite{Balanis_1997}.

%\begin{subequations}\label{eqn:inf_dipole_h_field}
%	\begin{equation}\label{eqn:hr_htheta}
%		H_r = H_\theta = 0
%	\end{equation}
%	\begin{equation}\label{eqn:hphi}
%		H_\phi = j \frac{k I_0 l \sin \theta}{4 \pi r}
%		\left[ 1 + \frac{1}{j k r} \right] e^{-j k r}
%	\end{equation}
%\end{subequations}
%
%
%\begin{subequations}\label{eqn:inf_dipole_e_field}
%	\begin{equation}\label{eqn:er}
%		E_r = \eta \frac{I_0 l \cos \theta}{2 \pi r^2} 
%		\left[ 1 + \frac{1}{jkr} \right] e^{-jkr}
%	\end{equation}
%	
%	\begin{equation}\label{eqn:etheta}
%		E_\theta = j \eta \frac{k I_0 l \sin \theta}{4 \pi r} 
%		\left[ 1 + \frac{1}{jkr} - \frac{1}{(kr)^2} \right] e^{-jkr}
%	\end{equation}
%
%	\begin{equation}\label{eqn:ephi}
%		E_\phi = 0
%	\end{equation}
%\end{subequations}

%The distance of $\lambda / 2\pi$, leading to $kr = 1$, is called radian distance, where $r$ is the distance to the antenna and $k$ is the free-space wave number. There, the real and imaginary part for both $H_\phi$ and $E_r$ is equal. Furthermore in $E_\theta$, only the $1/(jkr)$ term contributes to the field, because the other terms cancel each other out. The radian distance essentially marks a point of equal imaginary and stored power \cite{Balanis_1997}.
%
%In distances smaller than the radian distance $kr<1$, the fields are largely reactive and located in a reactive near-field region. There, the field strength decreases with the cube of the distance ($\propto 1/r^3$).  
%At distances beyond the radian distance (kr > 1), in the intermediate-field region, the field strength decreases with the square of the distance ($\propto 1/r^2$) \cite{Balanis_1997}. This behavior is visualized in \autoref{fig:energyfieldsregions}.
%
%The wave-number $k$ is dependent on the frequency. Therefore, when placing an infinitesimal dipole into a TEM cell, its coupling behavior depends on the distance and the field regions, that couple. For that reason, the $kr$ number shall be investigated over the whole frequency range for different distances.

