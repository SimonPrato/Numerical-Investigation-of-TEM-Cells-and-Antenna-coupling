\subsection{Radiated Field}\label{sec:rad_fields}
\FloatBarrier
\subsubsection{Field regions}

\begin{figure}[t]
	\centering
	\includegraphics[width=0.7\linewidth]{content/img/kr_over_distance}
	\caption{The behavior of Expression 1 and Expression 2 in (\ref{eqn:e_et_2}) is analyzed as a function of distance $r$. Here, $r$ is normalized to the radian distance $\lambda / 2\pi$, and the magnitudes of both expressions are scaled to unity at the radian distance to facilitate comparison \cite[p.~157]{Balanis_1997}.}
	\label{fig:kroverdistance}
\end{figure}

The field quantities $\mathbf{E}$ and $\mathbf{H}$ have been derived for an infinitesimal electric dipole in \cref{eqn:e_er,eqn:e_et,eqn:e_ep} and \cref{eqn:e_hr,eqn:e_hpt}, further for an infinitesimal magnetic dipole in \cref{eqn:m_hr,eqn:m_ht,eqn:m_hp} and \cref{eqn:m_ert,eqn:m_ep}. They are valid everywhere except for the source region \cite[p. 156]{Balanis_1997}.

The behavior of the fields depends on the distance $r$ from the dipole. This dependence becomes evident by investigating the terms $1/(jkr)$ and $1/(kr)^2$ appearing in \cref{eqn:e_er,eqn:e_et,eqn:e_ep} and \cref{eqn:e_hr,eqn:e_hpt} for the infinitesimal electric dipole. For clarity, these terms are highlighted in the expression for $E_\theta$ in \cref{eqn:e_et_2}, although they also partially appear in $E_r$ and $H_\phi$. They are denoted as Expression~1 and Expression~2, respectively:

\begin{equation}
	E_\theta = j\eta \frac{kI_0 d \sin \theta }{4\pi r}\left[ 1 + \underbrace{\frac{1}{jkr}}_{\text{Expression 1}} - \underbrace{\frac{1}{(kr)^2}}_{\text{Expression 2}} \right] \mathrm{e}^{-jkr}.
	\label{eqn:e_et_2}
\end{equation}



If the distance $r < \lambda/2\pi$ (or equivalently, $kr < 1$), then Expression~2 delivers the largest value in the brackets. Consequently, the energy stored in this region is predominantly imaginary. This region is referred to as the near-field region.

At distances $r > \lambda / 2\pi$ $(kr > 1)$, Expression~1 exceeds Expression~2 in magnitude, resulting in a larger real than imaginary part  of the energy. This region is referred to as the intermediate-field region. 

At larger distances $r \gg \lambda / 2\pi$ ($kr \gg 1$) the energy is predominantly real, reflecting radiated energy propagating outward. This region is referred to as the far-field region.

\begin{figure}[t]
	\centering
	\includegraphics[width=0.4\linewidth]{content/img/field_regions_antenna}
	\caption{Field regions of an antenna, here specifically a linear wire antenna, although they are applicable for any antenna with dimension $d$ \cite[p.~34]{Balanis_1997}.}
	\label{fig:fieldregionsantenna}
\end{figure}

At $r = \lambda/2\pi$ $(kr = 1)$, Expression 1 and Expression 2 attain equal magnitudes, a point referred to as the radian distance\cite[pp.~156-160]{Balanis_1997}. The radian distance thus represents a critical transition between field regions, marking a shift in field behavior. \autoref{fig:kroverdistance} shows the variation of Expression~1 and Expression~2 with respect to $r$. The analysis presented above for the field regions can similarly be applied to both small and infinitesimal magnetic dipoles.

For antennas that cannot be modeled as infinitesimal dipoles, the boundaries between reactive near-field, radiating near-field, and far-field regions shift compared to those of dipoles. The regions for such an antenna are shown in \autoref{fig:fieldregionsantenna}. Notably, in the far-field region, the antenna’s behavior closely approximates that of an infinitesimal electric dipole, while the radiating and reactive near-field regions exhibit different region sizes.

The far-field region starts at approximately $r_2$ and the radiating near-field at $r_1$, which are defined as

\begin{subequations}
	\begin{equation}
		r_1=0.62\sqrt{d^3/\lambda},
		\label{eqn:region_r1}
	\end{equation}
	\begin{equation}
	r_2=2d^2/\lambda.
	\label{eqn:region_r2}
	\end{equation}
\end{subequations}

Here, $d$ is the largest dimension of the antenna. In the case of the linear wire antenna, $d$ is the wire length \cite[pp. 165-170]{Balanis_1997}.

\FloatBarrier
\subsubsection{Energy densities and reactances}\label{sec:energy-densities-reactances}

The electric energy density $w_{\mathrm{e}}$ is defined as 
\begin{equation}
	w_{\mathrm{e}} = \frac{1}{2} \mathbf{E} \cdot \mathbf{D},
	\label{eqn:elec_density}
\end{equation}

while the magnetic energy density $w_{\mathrm{m}}$ is given by 

\begin{equation}
	w_{\mathrm{m}} = \frac{1}{2} \mathbf{B} \cdot \mathbf{H}.
	\label{eqn:mag_density}
\end{equation}

By applying the constitutive relations $\mathbf{D} = \epsilon \mathbf{E}$ and $\mathbf{B} = \mu \mathbf{H}$, and summing the contributions from (\ref{eqn:elec_density}) and (\ref{eqn:mag_density}), the total electromagnetic energy density $w_{\mathrm{em}}$ is obtained with \cite[p. 330]{Griffiths_2024} 

\begin{equation}
	w_\mathrm{em} = \frac{1}{2}\left(\underbrace{\vphantom{\frac{1}{\mu} B ^ 2}{\epsilon E^2}}_{{\text{Electric energy}\ w_e}} + \underbrace{\frac{1}{\mu} B ^ 2}_{{\text{Magnetic energy} \ w_m}}\right).
	\label{eqn:em_energy}
\end{equation}  

Integrating $w_\mathrm{em}$ over a given volume yields the total electromagnetic energy $W_\mathrm{em}$ contained within that region. Similarly, integrating $w_{\mathrm{e}}$ provides the total electric energy $W_{\mathrm{e}}$, and integrating $w_{\mathrm{m}}$ gives the total magnetic energy $W_{\mathrm{m}}$.

The reactance of an electrically short antenna is directly related to the electric and magnetic energy densities, $w_{\mathrm{e}}$ and $w_{\mathrm{m}}$, resulting from the antenna. The antenna’s equivalent inductance and capacitance can be determined using the relationships \cite[pp. 107, 328]{Griffiths_2024}

\begin{subequations}
	\begin{equation}
		L = 2\frac{W_m}{I^2},
		\label{eqn:l_m_energy}
	\end{equation}
	\begin{equation}
		C = 2 \frac{W_e}{V^2}.
		\label{eqn:c_e_energy}
	\end{equation}
\end{subequations}

