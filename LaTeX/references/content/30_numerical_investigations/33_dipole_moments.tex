\subsection{Dipole Moments}


\FloatBarrier
\subsubsection{Orientation and position in TEM Cell}

\autoref{fig:eyezcouplingcomparison} demonstrates the normalized output power of an electric dipole moment pointing in y-direction, and one in z-direction. This simulation only demonstrates the coupling behavior of the dipole moments over frequency, to explain the non-linear coupling of certain antennas. If dipole moments in certain positions and orientations couple with a different proportionality than the standard two dipole moments (ez and my), then the non-linear coupling may be explained that way.

\begin{figure}[h]
	\centering
	\includegraphics[width=1\linewidth]{content/img/ey_ez_coupling_comparison}
	\caption{Comparison of normalized output power of electric dipole moments}
	\label{fig:eyezcouplingcomparison}
\end{figure}

The electric dipole moment in z-direction $e_\mathrm{z}$ demonstrates the expected behavior: As the frequency rises, this dipole moment rises linearly and thus increases the output power quadratically. The electric dipole moment in y-direction $e_\mathrm{y}$ also increases linearly increase with frequency, but does not significantly change the output power for the low frequencies. However, as the frequency approaches the cut-off frequency of the next-higher order mode, the coupling rises significantly.

This simulation is repeated where the dipole moments are located at a height of $h=6\,\mathrm{mm}$, which is the dead center of the TEM cell, and $h=9\,\mathrm{mm}$, which is near the top wall of the TEM cell. The simulation results are similar for both cases. 

Most importantly, this simulation shows that the dipole moments have a relation to the frequency independent on their position. While their magnitude themselves do depend on the position, the relation to the frequency does not. 

\todo{Repeat Simulation for several other dipole moment positions and orientations?} 

\FloatBarrier
\subsubsection{Combining dipole moments with antennas}

\FloatBarrier
\subsubsection{Application of dipole moments}
As show in the previous simulations, antennas may be represented by dipole moments. This can be done in simulation models, which would otherwise be computationally too effortful. The dipole moments may be put into a shielded enclosure around a larger electronic system, as has been done in \cite{10274360}.

