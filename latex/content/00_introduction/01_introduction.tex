% !TeX root = document.tex

Over the past several years, electronic systems have exhibited a clear trend towards smaller physical dimensions and faster operating speeds. As such, they commonly contain small conducting structures carrying currents with high amplitude and frequency. These structures tend to radiate and are susceptible to electromagnetic radiation, behaving as antennas and causing electromagnetic compatibility (EMC) issues.

Designing an electronic system with EMC issues in mind minimizes additional cost and schedule delays caused by potential redesigns, and ensures the correct operation of the product in the presence of interference sources \cite[p. 64]{Paul_Scully_Steffka_2023}. Consequently, research focusing on all aspects of EMC is conducted regularly. This thesis aims to contribute to these ongoing investigations, specifically through analysis of the previously mentioned small antennas and their coupling behavior in TEM cells. TEM cells are included because they are useful for measuring electromagnetic emissions in approximate free-space conditions and have been widely used for small devices \cites{9153508,Koepke_1989,809846}.

Several studies have analyzed small antennas and devices in TEM cells \cites{Sreenivasiah_Chang_Ma_1981, 10274360}. Specifically, \cites{Kreindl_Bauernfeind_Weiss_Stockreiter_Kaltenbacher_2024, 10742020, Wilson_1981} implement electric and/or magnetic dipole moments to model the radiated fields of such antennas, which provide information about the electric and magnetic coupling with the TEM cell, respectively. The magnitudes of the dipole moments are found by measurements with the TEM cell \cite{Sreenivasiah_Chang_Ma_1981} or numerical analysis \cite{10742020}. This thesis treats the coupling behavior of small antennas modeled with dipole moments using the latter approach, namely numerical computation using the finite element method. The advantage of this approach is the absence of inaccuracies caused by the measurement setup or related uncertainties, allowing the analysis to focus on the underlying mechanics behind the coupling behavior.

The purpose of this thesis to fill the gap in understanding how the electric and magnetic dipole moments representing such antennas are formed and on which variables they depend. This knowledge assists in the EMC-compliant design of radiating structures in electronic devices, such that specific electric and magnetic coupling behavior is achieved. 

Furthermore, shielding materials in the presence of dipole moments are investigated. Replacing computationally expensive models of small radiating structures with dipole moments can significantly reduce the simulation time required to determine the coupling between the device and the TEM cell in the presence of shielding materials. 

To achieve these objectives, this thesis first presents the theoretical foundations of electric and magnetic dipole moments in \cref{sec:dipole-theory}. The behavior of electromagnetic waves generated by arbitrary sources in waveguides, such as the TEM cell, is then discussed in \cref{sec:guided-waves}. Further, methods to determine shielding effectiveness using the TEM cell are presented. A brief overview of the finite element method is provided in \cref{sec:simulations}. 

Subsequently, \cref{sec:num-inv} addresses the numerical modeling of antennas and the TEM cell and investigates the generation of electric and magnetic dipole moments for monopole and loop antennas using the theoretical framework developed earlier. This knowledge is applied to three additional antennas, whose analysis closely follows that of the monopole and loop antennas due to their predominantly inductive or capacitive characteristics, which emerge as the primary distinction in the antenna coupling behavior. Equivalent circuits to model capacitive and inductive antennas, together with the TEM cell and their coupling paths, are developed, from which the dipole moments can be investigated in more detail.

\cref{sec:shielding-sim} demonstrates the application of shielding materials in numerical simulations involving dipole moments and electrically small antennas. Lastly, \autoref{sec:conclusions} presents the conclusions and discussion derived from this thesis, along with potential directions for future research.

