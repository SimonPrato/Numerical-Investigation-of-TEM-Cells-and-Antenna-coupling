This thesis presents investigations of electrically small antennas and their coupling with a TEM cell using the finite element method. It further discusses applications of the framework created.

In this thesis, dipole moments equivalent to the electrically small antennas are calculated, whose magnitudes directly correlate with the electric and magnetic coupling of the antenna with the TEM cell. It finds, that the electric dipole moment correlates directly to the displacement current towards the septum, and the magnetic dipole moment to the voltage induced on the septum. An equivalent circuit model, both for capacitive and inductive antennas coupling to the TEM cell, is developed. 

The relation of different geometrical and electrical antenna parameters to the equivalent dipole moments is investigated. An increase in Q-factor or decrease in resonance frequency of the antenna has been found to increase non-linear dipole moments frequency-behavior. The electric dipole moment generated by an antenna increases primarily with its physical height, due to increased displacement currents toward the septum. The magnetic dipole moment increases with the loop area normal to the magnetic field intensity of a propagating mode in the TEM cell. If the loop is not closed, a magnetic dipole moment can still exist due to curling electric field intensities $\nabla \times \mathbf{E} \neq 0$ forming perpendicular to the magnetic field intensity. 

Further research could involve the measurement of such antennas with a real TEM cell, or the numerical analysis with other waveguides, such as the IC stripline. The framework in this thesis could be used to increase EMC of electronic systems containing electrically small, radiating structures, or represent them with dipole moments for less computational effort in complex simulation models.
