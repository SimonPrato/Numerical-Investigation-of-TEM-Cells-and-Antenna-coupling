\subsection{Radiated Field}

%
\subsection{Antenna Terminology}
Is this chapter necessary?
\subsection{Field regions}
Field regions, energy storage in each region, field distribution and auxiliary functions.

The region around an antenna may be divided into three sub-regions:
\begin{itemize}
	\item Reactive near-field region
	\item Radiating near-field or Fresnel region
	\item Far-field or Fraunhofer region
\end{itemize}

They are shown in \autoref{fig:fieldregionsantenna}. These regions are distinguished for most antennas using the wavelength $\lambda$ and the largest dimension of the antenna $D$. The nearest region to the antenna, the reactive near-field region, is a sphere with radius $R_1$, which is calculated with \autoref{eqn:region_r1}. It is characterized by the predominantly reactive fields in it. The next region, the radiating near-field region, is located around the first, within a sphere of radius $R_2$ calculated in \autoref{eqn:region_r2}. In there, the radiating fields dominate and the angular field distribution is dependent on the distance to the antenna. In the largest and last region, the far-field region, the angular field distribution is independent of the distance to the antenna \cite{Balanis_1997}.

\begin{subequations}
	\begin{equation}
		R_1=0.62\sqrt{D^3/\lambda}
		\label{eqn:region_r1}
	\end{equation}
	\begin{equation}
	R_2=2D^2/\lambda
	\label{eqn:region_r2}
	\end{equation}
\end{subequations}

\begin{figure}[h]
	\centering
	\includegraphics[width=0.4\linewidth]{content/10_theory/img/field_regions_antenna}
	\caption{Field regions of an antenna \cite{Balanis_1997}}
	\label{fig:fieldregionsantenna}
\end{figure}

The electrically short antennas located in the TEM cell interact with it mostly through the reactive near-field region. However, if the TEM cell is large and frequencies are high, the coupling occurs over the radiating near-field. There, the field distributions are different, hence the dipole moments couple differently over frequency to the cell. \todo[inline]{Is this even possible without reaching other modes behavior? And demonstrate the fields over frequency and distance, which changes differently in different regions}.

\subsubsection{Field regions of infinitesimal dipole}

Take an infinitesimal dipole as an antenna, for example. The fields are calculated through the auxiliary fields, as described earlier by \autoref{eqn:elec_and_mag_field_dipole}, which leads to \autoref{eqn:inf_dipole_h_field} for the magnetic field strength and \autoref{eqn:inf_dipole_e_field} for the electric field strength. Those equations are valid everywhere, except for the source point \cite{Balanis_1997}.

\begin{subequations}\label{eqn:inf_dipole_h_field}
	\begin{equation}\label{eqn:hr_htheta}
		H_r = H_\theta = 0
	\end{equation}
	\begin{equation}\label{eqn:hphi}
		H_\phi = j \frac{k I_0 l \sin \theta}{4 \pi r}
		\left[ 1 + \frac{1}{j k r} \right] e^{-j k r}
	\end{equation}
\end{subequations}


\begin{subequations}\label{eqn:inf_dipole_e_field}
	\begin{equation}\label{eqn:er}
		E_r = \eta \frac{I_0 l \cos \theta}{2 \pi r^2} 
		\left[ 1 + \frac{1}{jkr} \right] e^{-jkr}
	\end{equation}
	
	\begin{equation}\label{eqn:etheta}
		E_\theta = j \eta \frac{k I_0 l \sin \theta}{4 \pi r} 
		\left[ 1 + \frac{1}{jkr} - \frac{1}{(kr)^2} \right] e^{-jkr}
	\end{equation}

	\begin{equation}\label{eqn:ephi}
		E_\phi = 0
	\end{equation}
\end{subequations}

The distance of $\lambda / 2\pi$, or $kr = 1$, is called radian distance. There, the real and imaginary part for both $H_\phi$ and $E_r$ is equal. Furthermore in $E_\theta$, only the $1/(jkr)$ term contributes to the field, because the other terms cancel each other out. The radian distance essentially marks a point of equal imaginary and stored power \cite{Balanis_1997}.

In distances smaller than the radian distance $kr<1$, the fields are largely reactive and located in a reactive near-field region. There, the field strength decreases with the cube of the distance ($\propto 1/r^3$).  
At distances beyond the radian distance (kr > 1), in the intermediate-field region, the field strength decreases with the square of the distance ($\propto 1/r^2$) \cite{Balanis_1997}.

The wave-number $k$ is dependent on the frequency. Therefore, when placing an infinitesimal dipole into a TEM cell, its coupling behavior depends on the distance and the field regions, that couple. For that reason, the $kr$ number shall be investigated over the whole frequency range for different distances.





\subsection{Radiated Power}
Important for the Lorentz Reciprocity Part.

