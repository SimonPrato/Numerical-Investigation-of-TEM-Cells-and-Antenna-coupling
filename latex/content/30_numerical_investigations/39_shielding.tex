\section{Application of Shielding Techniques in TEM Cells}
\subsection{ASTM ES7-83 method}
\FloatBarrier

A model as described in \autoref{sec:astm} is used to determine the shielding effectiveness of nickel and zink according to the ASTM ES7-83 method discussed in \cref{sec:astm}. The TEM cell contains a shielding material sheet of thickness ranging from 1 to 40\,µm in the center of the TEM cell at $z=0$. A reference power $P_\mathrm{ref}=1\,\mathrm{W}$ is chosen, the load power $P_\mathrm{load}$ is numerically derived. Using \autoref{eqn:SE_power} yields the shielding effectiveness $SE_\mathrm{dB}$ of the material depending on thickness. The investigated frequency is 1\,GHz.

\todo[inline]{Compare results with other papers}

\begin{figure}
	\centering
	\includegraphics[width=0.5\linewidth]{content/img/se}
	\caption{Shielding effectiveness of a sheet of zinc and nickel versus the material thickness.}
	\label{fig:se}
\end{figure}

Metallic thin shields are normally removed from the computational domain and replaced by impedance network boundary conditions on the surfaces of the shielding material \cite{1430946}. Alternatively, the inside of the shielding material contains a fine mesh created with the adaptive meshing algorithm discussed in \cref{sec:simulations}, which is applied to these simulation models. 

\begin{table}[h]
	\centering

	{\renewcommand{\arraystretch}{1.2}
		\setlength{\tabcolsep}{12pt}
		\begin{tabular}{|l|c|c|c|}
			\hline
			Material & Rel. permittivity $\varepsilon_r$ & Rel. permeability $\mu_r$ & Conductivity $\sigma$ \\
			\hline\hline
			Zinc & $\approx 1$ & $\approx 1$ & $1.67 \times 10^7$\,S/m\\
			\hline
			Nickel & $\approx 1$ & $600$ & $1.45 \times 10^7$\,S/m \\
			\hline
	\end{tabular}}
	\caption{Electromagnetic Properties of Zinc and Nickel}
	\label{tab:materials}
\end{table}

\FloatBarrier
\subsection{Dual TEM cell}

\begin{figure}[h]
	\centering
	\includegraphics[width=0.5\linewidth]{content/img/empty-aperture-coupling}
	\caption{Coupling of port 1 to ports 3 and 4 with empty square aperture}
	\label{fig:empty-aperture-coupling}
\end{figure}

Empty square aperture with side length of $d=11.2\,\mathrm{mm}$ used. Simulation model leans on measurement setup in \cite{4091811}.


\subsection{Shielding Antennas}

\todo[inline]{shielding antennas and investigating field distribution will be done here. Some tilt in shielding material would be interesting to investigate}

\subsection{Shielding Equivalent Dipole Moments}

\todo[inline]{Check latex comments, which contain some good ideas. TODO Rest of section}

%\todo[inline]{Rest of shielding material section still TODO}
%\subsection{Shielding effectiveness of graphite}
%
%The reference power $P_\mathrm{ref}$ has been set to 1\,W. Using \autoref{eqn:load_power} and the S-parameters from the simulation results, $P_\mathrm{load}$ may be determined. \autoref{fig:se_graphite} demonstrates the shielding effectiveness of graphite in dB $SE_\mathrm{dB}$ over the shielding material thickness. The solution frequency is 500\,MHz. A frequency sweep shows that the reflection coefficient $S_{11}$ does not depend much on the frequency. 
%
%\begin{figure}[h]
%	\centering
%	\includegraphics[width=1\linewidth]{content/img/se_graphite.png}
%	\caption{Shielding effectiveness of graphite}
%	\label{fig:se_graphite}
%\end{figure}
%
%The components of $SE_\mathrm{dB}$ are determined according to \autoref{eqn:se_rereflections}. 
%
%\subsection{Shield effectiveness of FR4}
%
%The FR4 has a relative permittivity of $\epsilon_r=4.4$. According to \autoref{eqn:rel_wave_imp}, the relative wave impedance is $Z=0.476$. This leads to a reflection coefficient of $R=-0.355$ by \autoref{eqn:reflection_coefficient_plane_dielectric}.
%
%
%The reflection coefficient $|S_{11}|=0.045$.
%
%\subsection{Dual TEM Cell}
%
%A simulation setup of a dual TEM cell is created. A rectangular aperture with a side length of $l=5\,\mathrm{cm}$, inspired by \cite{Wilson_1981}, connects both TEM cells. One waveport 1, as in \autoref{fig:dual_tem_cell}, is excited with a power of $P=1\,\mathrm{W}$. The simulation is conducted, leaving the aperture open. A second one determines the coupling of the waveports, when the aperture is filled with a graphite sheet with a thickeness of $t=50$\,µm.
%
%At a frequency of $f=500\,\mathrm{MHz}$, the coupling between waveport 1 to the waveports 3 and 4 of the receiving TEM cell is shown in \autoref{tab:dual_tem_fwd_trans}. Only one frequency point is investigated, as the results stay roughly constant over the inspected frequency range from 100\,MHz to 1\,GHz. 
%
%
%\begin{table}
%	\centering
%	\begin{tabular}{|c|c|c|}
%		\hline
%		Forward transmission coefficient & Empty aperture & aperture filled with FR408\\\hline\hline
%		Waveport 1 to 3 $S_{13}$ & -83.80\,dB, -144.96° & -85.27\,dB, -155.79°\\\hline
%		Waveport 1 to 4 $S_{14}$ & -90.31\,dB, -144.96° & -87.14\,dB, 25.00°\\\hline
%	\end{tabular}
%	\caption{Forward transmission coefficients}
%	\label{tab:dual_tem_fwd_trans}
%\end{table}
%\todo{Why -8dB difference in empty aperture? Explained in \cite{Wilson_1981}}
%
%Using \autoref{eqn:se_dual_cell_e} and \autoref{eqn:se_dual_cell_m} leads to the shielding effectiveness for electric coupling $SE_\mathrm{dB}^\mathrm{e}=19.07\,\mathrm{dB}$ and magnetic coupling $SE_\mathrm{dB}^\mathrm{m}=-9.22\,\mathrm{dB}$. \todo{negative SE possible? Redo Simulations with finer Mesh around aperture} To get the sum $P_\mathrm{sum}$ and difference $P_\mathrm{diff}$ of powers, the phase of the signals have to be considered. With unit input power at the transmitting TEM cell, \autoref{eqn:s_param_to_power_sum} and \autoref{eqn:s_param_to_power_diff} are used for this purpose \cite{Sreenivasiah_Chang_Ma_1981}. 
%
%\begin{subequations}
%	\begin{equation}
%		P_\mathrm{sum} = (S_{13} + S_{14})(S_{13} + S_{14})^*
%		\label{eqn:s_param_to_power_sum}
%	\end{equation}
%	\begin{equation}
%		P_\mathrm{diff}= (S_{13} - S_{14})(S_{13} - S_{14})^*
%		\label{eqn:s_param_to_power_diff}
%	\end{equation}
%\end{subequations}
%
%Indicated by the phase shift of roughly 180°, the coupling between the TEM cells occur mainly due to magnetic dipoles. Due to the relative permittivity of $\epsilon _\mathrm{r}=3.66$ and the relative permeability of $\mu_r\approx 1$ of the shielding material, the magnetic fields dominate. This leads to a energy transfer mainly due to magnetic dipole moments\todo{One port receives overall more power due to the material. Is it because of the magnetic/electric dipoles in it? Check mesh around the small aperture.} The overall shielding effectiveness $SE_\mathrm{dB}=$ \autoref{eqn:dual_tem_cell_tot_power}.
%
%\begin{equation}
%	P_\mathrm{total}=|S_{13}|^2+|S_{14}|^2
%	\label{eqn:dual_tem_cell_tot_power}
%\end{equation}

% Show which dipole moments are affected by an offset in z-direction, and which ones are not.
