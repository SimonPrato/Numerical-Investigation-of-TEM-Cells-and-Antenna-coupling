% !TeX root = Documentation.tex

In recent years, electronic systems have demonstrated a clear trend toward reduced physical dimensions and increased operating speeds, consequently, higher frequencies are used. As a result, these systems often contain small conducting structures that carry currents and voltages with high amplitudes and frequencies. These structures tend to radiate and are susceptible to electromagnetic radiation, behaving as antennas and causing electromagnetic compatibility (EMC) issues.

Taking EMC into account during the design of electronic systems helps to minimize additional costs and schedule delays that may arise from potential redesigns. Furthermore, it ensures that the product operates reliably when exposed to interference from external sources \cite[p.~64]{Paul_Scully_Steffka_2023}. Consequently, research focusing on all aspects of EMC is conducted regularly. This thesis aims to contribute to these ongoing investigations, specifically through analysis of the previously mentioned small antennas and their coupling behavior in TEM cells. TEM cells are included because they provide a standardized method for measuring electromagnetic emissions under approximate free-space conditions and have been widely used for testing small devices \cites{9153508,Koepke_1989,809846}.

Several studies have analyzed the coupling behavior of small antennas and devices with TEM cells \cites{Sreenivasiah_Chang_Ma_1981, 10274360}. Specifically, \cites{Kreindl_Bauernfeind_Weiss_Stockreiter_Kaltenbacher_2024, 10742020, Wilson_1981} implement electric and/or magnetic dipole moments to model the radiated fields of such antennas, which provide information about the electric and magnetic coupling with the TEM cell, respectively. The magnitudes of the dipole moments are found by measurements with the TEM cell \cite{Sreenivasiah_Chang_Ma_1981} or numerical analysis \cite{10742020}. This thesis treats the coupling behavior of small antennas modeled with dipole moments using the latter approach, namely numerical computation using the finite element method. The advantage of this approach is the absence of inaccuracies caused by the measurement setup or related uncertainties, allowing the analysis to focus on the underlying mechanics behind the coupling behavior.

This thesis aims to explain how the electric and magnetic dipole moments of antennas are created and what factors affect them. Understanding this helps design electronic devices that meet EMC requirements and achieve specific coupling behaviors. Additionally, replacing the small antennas with their equivalent dipole moments significantly reduces computational effort, which is particularly advantageous when dealing with large computational domains.

Furthermore, this thesis investigates the shielding efficiency of different materials in the presence of dipole moments. The performance of the shielding material with respect to the electric and magnetic coupling behavior of the antennas, as reflected by the dipole moments, is investigated. The results assist in the selection of appropriate shielding material to effectively reduce emissions produced by the antennas. 

To achieve these objectives, this thesis first presents the theoretical foundations of electric and magnetic dipole moments in \cref{sec:dipole-theory}. The behavior of electromagnetic waves generated by arbitrary sources in waveguides, specifically the TEM cell, is then discussed in \cref{sec:guided-waves}. Further, background information of electromagnetic shielding and methods to determine shielding effectiveness using the TEM cell are presented. A brief overview of the finite element method is provided in \cref{sec:simulations}. 

Subsequently, \cref{sec:num-inv} addresses the numerical modeling of antennas and the TEM cell and investigates the generation of electric and magnetic dipole moments for monopole and loop antennas using the theoretical framework developed earlier. This knowledge is applied to three additional antennas, whose analysis delivers results closely related to that of the monopole and loop antennas due to their shared predominantly inductive or capacitive characteristics, which emerge as the primary distinction in the antenna coupling behavior. Equivalent circuits to model capacitive and inductive antennas, together with the TEM cell and their coupling paths, are developed, from which the dipole moments can be investigated in more detail.

\cref{sec:shielding-sim} demonstrates the application of shielding materials in numerical simulations involving dipole moments and electrically small antennas. Lastly, \autoref{sec:conclusions} presents the conclusions and discussion derived from this thesis, along with potential directions for future research.

