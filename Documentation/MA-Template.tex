\documentclass[11pt]{article}
%\usepackage{ngerman}
\usepackage[german]{babel}
\usepackage{amsmath,amssymb,amstext}
\usepackage{graphicx}
\usepackage{parskip}
\usepackage{float}
\usepackage{tabularx}
%\usepackage{subtextbf}
\usepackage{pdfpages}
\usepackage{url}
%\usepackage{cite}
\usepackage{array}
\usepackage{multirow}
\usepackage{hyperref}
\usepackage{biblatex}


 
\bibliography{sample}
\bibliographystyle{plain}
\setlength{\textwidth}{15 true cm}
\setlength{\textheight}{22 true cm}
\oddsidemargin  0.5 cm
\evensidemargin 0.5 cm
\topmargin      0 cm

\renewcommand{\textfraction}{.2}
\renewcommand{\floatpagefraction}{.8}

\begin{document}
%\frontmatter
\pagenumbering{roman}

% Include titlepage
\begin{titlepage}	
	{\sffamily		
		\begin{center}			
			\includegraphics[width=30mm]{images/TU_Graz_Logo.png}
			
			\vfill\vfill\vfill
			\vfill\vfill\vfill
			
			{Simon Prato}
			% Author with existing titles
			
			\vfill\vfill\vfill
			
			{\LARGE\bfseries{Entwicklung und Simulation eines RF-Multiplexers}}
			% Title of the thesis			
			
			\vfill\vfill\vfill
			\vfill\vfill\vfill			
			
			{\bfseries\large{Bachelorarbeit}}
			
			{Studiengang: {Elektrotechnik}}
						
			\vfill\vfill\vfill			
			
			eingereicht bei
			
			\vfill
			
			{\bfseries\large{Technische Universität Graz}}			
			
			\vfill\vfill\vfill			
			
			Betreuer
			
			{Dr. Peter Söser}
			
			\vfill
			
			\vfill
			
			\includegraphics[width=30mm]{images/IFE_Logo.png}
			
			%% OPTIONAL: second supervisor/name of the faculty, etc.
					
			\vfill\vfill\vfill
					
			{Graz}, {August}~{2024}
			
		\end{center}
	}%% end sffamily
\end{titlepage}

\newpage

% Kurzfassung
\iftrue
\cleardoublepage
\setcounter{page}{2}
\vspace*{2.2 cm}
{\Large
\noindent
{\bf Kurzfassung}} \\
\vspace*{0.3 cm}

\noindent

% Abstract
\cleardoublepage
\fi

\selectlanguage{german}
% Inhaltsverzeichnis
	\tableofcontents  

% Tabellenverzeichnis
% optional
\newpage
	\listoffigures 

% Abbildungsverzeichnis
% optinal
\newpage
	\listoftables
	
\cleardoublepage

%\mainmatter
\pagestyle{headings}
\pagenumbering{arabic}

\section{Einleitung}


\newpage
%Die Ansteuerung über das firmeninterne Netzwerk funktioniert fehlerlos. Die Maße der Leiterplatte passen. Der Multiplexer dient nun als Erweiterung eines Prüfsystems der Firma Infineon.
\iffalse
\subsection{Zusammenfassung}

\fi
\cleardoublepage
\printbibliography

\end{document}
