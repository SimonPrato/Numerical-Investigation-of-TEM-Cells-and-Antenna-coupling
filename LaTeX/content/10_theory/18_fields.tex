\subsection{Radiated Field}

%
\subsection{Antenna Terminology}
Is this chapter necessary?
\subsection{Field regions}
Field regions, energy storage in each region, field distribution and auxiliary functions.

The region around an antenna may be divided into three sub-regions:
\begin{itemize}
	\item Reactive near-field region
	\item Radiating near-field or Fresnel region
	\item Far-field or Fraunhofer region
\end{itemize}

They are shown in \autoref{fig:fieldregionsantenna}. These regions are distinguished for most antennas using the wavelength $\lambda$ and the largest dimension of the antenna $D$. The nearest region to the antenna, the reactive near-field region, is a sphere with radius $R_1$, which is calculated with \autoref{eqn:region_r1}. It is characterized by the predominantly reactive fields in it. The next region, the radiating near-field region, is located around the first, within a sphere of radius $R_2$ calculated in \autoref{eqn:region_r2}. In there, the radiating fields dominate and the angular field distribution is dependent on the distance to the antenna. In the largest and last region, the far-field region, the angular field distribution is independent of the distance to the antenna \cite{Balanis_1997}.

\begin{subequations}
	\begin{equation}
		R_1=0.62\sqrt{D^3/\lambda}
		\label{eqn:region_r1}
	\end{equation}
	\begin{equation}
	R_2=2D^2/\lambda
	\label{eqn:region_r2}
	\end{equation}
\end{subequations}

\begin{figure}[h]
	\centering
	\includegraphics[width=0.7\linewidth]{content/10_theory/img/field_regions_antenna}
	\caption{Field regions of an antenna \cite{Balanis_1997}}
	\label{fig:fieldregionsantenna}
\end{figure}

The electrically short antennas located in the TEM cell interact with it mostly through the reactive near-field region. However, if the TEM cell is large and frequencies are high, the coupling occurs over the radiating near-field. There, the field distributions are different, hence the dipole moments couple differently over frequency to the cell. \todo[inline]{Is this even possible without reaching other modes behavior? And demonstrate the fields over frequency and distance, which changes differently in different regions}.



\subsection{Radiated Power}
Important for the Lorentz Reciprocity Part.

